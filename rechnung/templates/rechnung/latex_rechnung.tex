\documentclass[a4paper,12pt,top=1.5cm,bottom=1.5cm]{article}
\usepackage[ngerman]{babel}
\usepackage[utf8]{inputenc}
\usepackage{eurosym}
\usepackage{geometry}
\usepackage{etoolbox}
\usepackage{booktabs}
\usepackage{amsmath}
\geometry{margin=2cm}
%\pagestyle{empty}


\begin{document}
%Firma oben rechts
\raggedleft{\parbox[c]{10cm}{
    \raggedleft \bf Verein zur Studienförderung der Fachschaft Mathematik/Physik/Informatik e. V.}}
\\[8ex]

%Adressfeld
\raggedright{\parbox[c]{8cm}{
    \raggedright \scriptsize Verein zur Studienförderung der FS MPI e. V., Boltzmannstraße 3, 85748 Garching\\[1ex] \hrule}}\\[3ex]
\raggedright
\notblank{ {{ rechnung.kunde.organisation }} }{ {{ rechnung.kunde.organisation }} \\}{}
\notblank{ {{ rechnung.kunde.suborganisation }} }{ {{ rechnung.kunde.suborganisation }} \\}{}
\notblank{ {{ rechnung.kunde.vorname }}{{ rechnung.kunde.name }} }{ {{ rechnung.kunde.vorname }} {{ rechnung.kunde.name}} \\}{}
{{ rechnung.kunde.strasse }}\\
{{ rechnung.kunde.plz }} {{ rechnung.kunde.stadt }}\\[3ex]

%Infofeld
\hfill{\parbox[c]{7.5cm}{
    \begin{tabular}{ll}
        Kundennummer: & {{ rechnung.kunde.knr }} \\[-0.5ex]
        Rechnungsnummer: & {{ rechnung.rnr }} \\[-0.5ex]
        Rechnungsdatum: & {{ rechnung.rdatum }} \\[-0.5ex]
        Lieferdatum: & {{ rechnung.ldatum }} \\[2ex]
    \end{tabular}
}}

%Text
\large{\bf Rechnung}\\[3ex]
\normalsize
\if{rechnung.kunde.anrede='w'}
Sehr geehrte Frau
\else{Sehr geehrter Herr}
\fi
{{ rechnung.kunde.name }},\\[2ex]

{{ rechnung.einleitung }}\\[2ex]

\hfil
\begin{tabular}{llll}
    \toprule
    Anzahl & Bezeichnung & Einzelpreis & Mwst-Satz\\
    \midrule
    \forall{ {{ posten }} \in {{ rechnung.posten.all }} }
        a & b & c & d\\
%        anzahl & {{ posten.name }} & {{ posten.einzelpreis }} & {{ posten.mwst }}\\
    \endfor
    \bottomrule
\end{tabular}
\end{document}
