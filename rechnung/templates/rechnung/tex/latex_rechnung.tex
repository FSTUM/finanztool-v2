
    
    \documentclass[a4paper,12pt,top=1.5cm,bottom=1.5cm]{article}
    \usepackage[ngerman]{babel}
    \usepackage[utf8]{inputenc}
    \usepackage{eurosym}
    \usepackage{geometry}
    \usepackage{etoolbox}
    \usepackage{booktabs}
    \usepackage{amsmath}
    \usepackage{fancyhdr}
    \usepackage{longtable}
    \usepackage{tabularx}
    \usepackage{multirow}
    \usepackage[absolute]{textpos}
    \usepackage{graphicx}
    \geometry{margin=2cm}
    \pagestyle{fancy}
    \renewcommand{\headrulewidth}{0pt}
    \textheight = 670pt
    \textwidth = 18cm
    \newcolumntype{L}[1]{>{\raggedright\arraybackslash}p{#1}} % linksbündig mit Breitenangabe
    \newcolumntype{C}[1]{>{\centering\arraybackslash}p{#1}} % zentriert mit Breitenangabe
    \newcolumntype{R}[1]{>{\raggedleft\arraybackslash}p{#1}} % rechtsbündig mit Breitenangabe

    \begin{document}
    %Folding marks%%%%%%%%%%%%%%%%%%%%%%%%%%%%%%%%%%%%%%%%%%%%%%%%%%%%%%%%%%%%%%%%
    \textblockorigin{-3mm}{0mm}
    \setlength{\TPHorizModule}{1mm}\setlength{\TPVertModule}{1mm}
    \begin{textblock}{1}(0,98) \rule[0.5ex]{2mm}{1pt} \end{textblock}
    \begin{textblock}{1}(0,204) \rule[0.5ex]{2mm}{1pt} \end{textblock}

    %Header%%%%%%%%%%%%%%%%%%%%%%%%%%%%%%%%%%%%%%%%%%%%%%%%%%%%%%%%%%%%%%%%%%%%%%%
    \lhead{}
    \chead{}
    \rhead{\bf
        Verein zur Studienförderung der Fachschaft\\
        Mathematik/Physik/Informatik e. V.}
    \headsep = 2.8 cm

    %Footer%%%%%%%%%%%%%%%%%%%%%%%%%%%%%%%%%%%%%%%%%%%%%%%%%%%%%%%%%%%%%%%%%%%%%%%
    \lfoot{}
    \cfoot{
        \parbox[c][2.5cm][t]{5.5cm}{
            \scriptsize
            Studienförderung der Fachschaft\\
            Mathematik/Physik/Informatik e. V.\\
            Boltzmannstraße 3\\
            85748 Garching}
        \parbox[c][2.5cm][t]{3.5cm}{
            \scriptsize
            Kreissparkasse München-\\
            Starnberg-Ebersberg\\
            Konto 027 003 722\\
            BLZ 702 501 50}
        \parbox[c][2.5cm][t]{4.5cm}{
            {\tiny IBAN \& BIC:\\}
            \scriptsize
            DE68 7025 0150 0027 0037 22\\
            BYLADEM1KMS\\[0.75ex]
            USt-ID: DE129519467}
        \parbox[c][2.5cm][t]{3cm}{
            \scriptsize
            T 089/289 18545\\
            F 089/289 18546\\
            finanz@fs.tum.de\\
            https://mpi.fs.tum.de}
    }
    \rfoot{}
    \footskip = -0.2cm


    %Adressfeld%%%%%%%%%%%%%%%%%%%%%%%%%%%%%%%%%%%%%%%%%%%%%%%%%%%%%%%%%%%%%%%%%%%
    \raggedright{\parbox[c]{8cm}{
        \raggedright \scriptsize
        Verein zur Studienförderung der FS MPI e. V.,
        Boltzmannstraße 3, 85748 Garching\\[1ex] \hrule}}\\[2ex]
    \raggedright{\parbox[c]{8cm}{\footnotesize
    \notblank{ {{ rechnung.kunde.organisation|default_if_none:"" | latex_escape }} }
        { {{ rechnung.kunde.organisation | latex_escape }}\\[0.5ex]}{}
    \notblank{ {{ rechnung.kunde.suborganisation|default_if_none:"" | latex_escape }} }
        { {{ rechnung.kunde.suborganisation | latex_escape }} \\[0.2ex]}{}
    \notblank{ {{ rechnung.kunde.name|default_if_none:"" | latex_escape }} }
        { {{ rechnung.kunde.titel|default_if_none:"" | latex_escape }} {{ rechnung.kunde.vorname|default_if_none:"" | latex_escape }} {{ rechnung.kunde.name | latex_escape }}  \\[0.2ex]}{}
    {{ rechnung.kunde.strasse | latex_escape }} \\[0.2ex]
    \ifnum\pdfstrcmp{ {{ rechnung.kunde.land | latex_escape }} }{ Deutschland }=0
        {{ rechnung.kunde.plz | latex_escape }} {{ rechnung.kunde.stadt | latex_escape }}\\[3ex]
    \else
        {{ rechnung.kunde.plz | latex_escape }} {{ rechnung.kunde.stadt | latex_escape }}\\[0.2ex]
        {{ rechnung.kunde.land | latex_escape }} \\[3ex]
    \fi
    }}


    %Infofeld%%%%%%%%%%%%%%%%%%%%%%%%%%%%%%%%%%%%%%%%%%%%%%%%%%%%%%%%%%%%%%%%%%%%%
    \hfill{\parbox[c]{7.5cm}{
        \vspace{-150px}
        \begin{tabular}{ll}
            
                Zahlungsdetails als & \multirow{6}{*}{ \includegraphics[width=.15\textwidth]{ {{rechnung.epc_qr_code.path}} } }\\
                EPC-QR-Code:&                \\\\\\\\\\
            
            Kundennummer: &{{ rechnung.kunde.knr|unlocalize }} \\[-0.5ex]
            Rechnungsnummer: & {{ rechnung.rnr_string }} \\[-0.5ex]
            
                \ifnum\pdfstrcmp{ {{ rechnung.ldatum }} }{ None }=0
                    Datum: & {{ mahnung.mdatum }} \\[7.5ex]
                \else
                    Lieferdatum: & {{ rechnung.ldatum }} \\[-0.5ex]
                    Datum: & {{ mahnung.mdatum }} \\[2ex]
                \fi
            
                \ifnum\pdfstrcmp{ {{ rechnung.ldatum }} }{ None }=0
                    Rechnungsdatum: & {{ rechnung.rdatum }} \\[7.5ex]
                \else
                    Lieferdatum: & {{ rechnung.ldatum }} \\[-0.5ex]
                    Rechnungsdatum: & {{ rechnung.rdatum }} \\[2ex]
                \fi
            
        \end{tabular}
    }}

    %Text%%%%%%%%%%%%%%%%%%%%%%%%%%%%%%%%%%%%%%%%%%%%%%%%%%%%%%%%%%%%%%%%%%%%%%%%%
    
        \large{\bf {{ mahnung.wievielte }}. Mahnung}\\[3ex]
    
        \large{\bf Rechnung}\\[3ex]
    

    \normalsize
    \ifnum\pdfstrcmp{ {{ rechnung.kunde.anrede | latex_escape }} }{ w }=0
        Sehr geehrte Frau {{ rechnung.kunde.titel|default_if_none:"" | latex_escape }} {{ rechnung.kunde.name | latex_escape }},\\[2ex]
    \else
        \ifnum\pdfstrcmp{ {{ rechnung.kunde.anrede | latex_escape }} }{ m }=0
            Sehr geehrter Herr {{ rechnung.kunde.titel|default_if_none:"" | latex_escape }} {{ rechnung.kunde.name | latex_escape }},\\[2ex]
        \else
            Sehr geehrte Damen und Herren,\\[2ex]
        \fi
    \fi

    
        {{ mahnung.einleitung | latex_escape }} \ \\[3ex]
    
        {{ rechnung.einleitung | latex_escape }} \ \\[3ex]
    

    
        %Posten%%%%%%%%%%%%%%%%%%%%%%%%%%%%%%%%%%%%%%%%%%%%%%%%%%%%%%%%%%%%%%%%%%%%%%%
        \setlength\LTleft{0pt}
        \setlength\LTright{0pt}
        \begin{longtable}{@{\extracolsep{\fill}}R{2cm}L{7.5cm}R{3cm}R{2.75cm}C{0.85cm}}
        %\begin{longtable}{@{\extracolsep{\fill}}rlp{2cm}rp{0.75cm}}
            \toprule
            Anzahl & Bezeichnung & Einzelpreis & Summe {\small (netto)} & Mwst\\
            \midrule
            \endhead
            
                {{ posten.anzahl }} &
                {{ posten.name | latex_escape }} &
                {{ posten.einzelpreis }} \euro &
                {{ posten.summenettogerundet }} \euro &
                \ifnum\pdfstrcmp{ {{ posten.mwst }} }{ 7 }=0
                    B\\[1.5ex]
                \else
                    \ifnum\pdfstrcmp{ {{ posten.mwst }} }{ 19 }=0
                        A\\[1.5ex]
                    \else
                        \\[1.5ex]
                    \fi
                \fi
            
            \bottomrule
             & Zwischensumme: & & {{ rechnung.zwischensumme }} \euro &\\[1ex]
             & {\footnotesize\ Zzgl. Mwst 19 \% (A):} & & {\footnotesize{{ rechnung.summe_mwst_19 }} \euro} &\\
             & {\footnotesize\ Zzgl. Mwst 7 \% (B):} & & {\footnotesize{{ rechnung.summe_mwst_7 }} \euro} &\\[1ex]
             & {\bf Gesamt:} & & {\bf {{ rechnung.gesamtsumme }} \euro} &\\
            \bottomrule
            \bottomrule\\[-1ex]
        \end{longtable}

        %Endtext%%%%%%%%%%%%%%%%%%%%%%%%%%%%%%%%%%%%%%%%%%%%%%%%%%%%%%%%%%%%%%%%%%%%%%
        Bitte überweisen Sie den Rechnungsbetrag von {\bf {{ rechnung.gesamtsumme }}} \euro\ ohne Abzug bis zum
        {{ rechnung.fdatum }} auf das unten angegebene Konto mit Verwendungszweck
                \textbf{\texttt{ '{{ rechnung.rnr_string }}'. }}\\[2ex]

    
        %Mahnungen%%%%%%%%%%%%%%%%%%%%%%%%%%%%%%%%%%%%%%%%%%%%%%%%%%%%%%%%%%%%%%%%%%%%
        \setlength\LTleft{0pt}
        \setlength\LTright{0pt}
        \begin{longtable}{@{\extracolsep{\fill}}L{2.5cm}L{3cm}L{3cm}R{7cm}}
            \toprule
            Posten & Datum & Fälligkeit am & Betrag\\
            \bottomrule\\[-1ex]
            \endhead
            {{ rechnung.rnr_string }} &
            {{ rechnung.rdatum }} &
            {{ rechnung.fdatum }} &
            {{ rechnung.gesamtsumme }} \euro\\
            \midrule
            
                
                    {{ vorherige.wievielte }}. Mahnung &
                    {{ vorherige.mdatum }} &
                    {{ vorherige.mfdatum }} &
                    {{ vorherige.gebuehr }} \euro\\
                    \midrule
                
            
            {{ mahnung.wievielte }}. Mahnung &
            {{ mahnung.mdatum }} &
            {{ mahnung.mfdatum }} &
            {{ mahnung.gebuehr }} \euro\\
            \bottomrule
            \bottomrule\\[-0.5ex]
            {\bf Summe} & & & {\bf {{ mahnung.mahnsumme }} \euro}\\
        \end{longtable}

        %Endtext%%%%%%%%%%%%%%%%%%%%%%%%%%%%%%%%%%%%%%%%%%%%%%%%%%%%%%%%%%%%%%%%%%%%%%
        Bitte beachten Sie, dass das Zahlungsziel bereits überschritten ist.
        
            % Ja, hier ist leider Text doppelt, sieht aber sonst komisch im LaTex aus
            Wir bitten Sie, zur {\bf Vermeidung gerichtlicher Schritte}, die wir
            hiermit widrigenfalls bereits ankündigen, um Anweisung der im Verzug
            befindlichen Positionen bis zum
        %
            Wir bitten Sie um Prüfung und Anweisung der im Verzug befindlichen
            Positionen bis zum
        %
            {\bf {{ mahnung.mfdatum }}} auf das unten angegebene Konto mit Verwendungszweck
            {\bf {{ rechnung.rnr_string }}}.\\
            Sollte die Zahlung zwischenzeitlich erfolgt sein, betrachten Sie bitte
            dieses Schreiben als gegenstandslos. Sollte es Einwände gegen die
            Regulierung geben, bitten wir um kurze Information.\\[2ex]

    

    Bei Rückfragen stehen wir Ihnen gerne zur Verfügung!


    \end{document}

